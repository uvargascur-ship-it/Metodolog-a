% =========================================================================
%      MANUAL Y PLANTILLA DE LATEX PARA NORMAS APA 7ma EDICIÓN
%                   Adaptado para la UNMSM
% =========================================================================

\documentclass[12pt, a4paper]{article}

% -------------------------------------------------------------------------
% 1. CODIFICACIÓN E IDIOMA
% -------------------------------------------------------------------------
\usepackage[utf8]{inputenc} % Permite tildes y ñ directamente
\usepackage[T1]{fontenc}    % Correcta codificación de salida
\usepackage[spanish, es-tabla, es-nodecimaldot]{babel}
% es-tabla: Cambia la etiqueta "Cuadro" por "Tabla" (Norma APA).
% es-nodecimaldot: Usa punto (.) para decimales (10.5) en vez de coma (10,5).

% -------------------------------------------------------------------------
% 2. FORMATO DE PÁGINA Y TEXTO (APA BÁSICO)
% -------------------------------------------------------------------------
% Márgenes: 1 pulgada (2.54 cm) en todos los lados.
\usepackage[margin=2.54cm]{geometry}

% Fuente: Times New Roman (Texto y Matemáticas).
\usepackage{mathptmx}

% Espaciado: Doble espacio estricto en todo el documento.
\usepackage{setspace}
\doublespacing

% Sangría: 1.27 cm (1/2 pulgada) en la primera línea de cada párrafo.
\setlength{\parindent}{1.27cm}

% -------------------------------------------------------------------------
% 3. ENCABEZADO Y NUMERACIÓN DE PÁGINA
% -------------------------------------------------------------------------
\usepackage{fancyhdr}
\setlength{\headheight}{15.2pt} % Ajuste técnico para evitar errores
\pagestyle{fancy}
\fancyhf{} % Limpia encabezados y pies por defecto

% LADO IZQUIERDO: Título Corto (Running Head)
% Nota: Debe ser en MAYÚSCULAS y máximo 50 caracteres.
\fancyhead[L]{DETERMINANTES DE LA VICTIMIZACIÓN EN HUAYCÁN}

% LADO DERECHO: Número de página
\fancyhead[R]{\thepage}

% Elimina la línea horizontal del encabezado (APA no usa líneas decorativas)
\renewcommand{\headrulewidth}{0pt}

% -------------------------------------------------------------------------
% 4. JERARQUÍA DE TÍTULOS (APA 7 - 5 NIVELES)
% -------------------------------------------------------------------------
\usepackage{titlesec}

% Nivel 1: Centrado, Negrita (Ej. \section{Introducción})
\titleformat{\section}[block]{\centering\normalfont\normalsize\bfseries}{}{0em}{}

% Nivel 2: Izquierda, Negrita (Ej. \subsection{Subtema})
\titleformat{\subsection}[block]{\raggedright\normalfont\normalsize\bfseries}{}{0em}{}

% Nivel 3: Izquierda, Negrita, Cursiva (Ej. \subsubsection{Detalle})
\titleformat{\subsubsection}[block]{\raggedright\normalfont\normalsize\bfseries\itshape}{}{0em}{}

% Nivel 4: Sangría, Negrita, Punto final. Texto sigue en la línea.
\titleformat{\paragraph}[runin]{\normalfont\normalsize\bfseries}{}{0em}{}[.]
\titlespacing{\paragraph}{1.27cm}{0pt}{1em}

% Nivel 5: Sangría, Negrita, Cursiva, Punto final. Texto sigue en la línea.
\titleformat{\subparagraph}[runin]{\normalfont\normalsize\bfseries\itshape}{}{0em}{}[.]
\titlespacing{\subparagraph}{1.27cm}{0pt}{1em}

% -------------------------------------------------------------------------
% 5. CONFIGURACIÓN DE TABLAS Y FIGURAS (ESTILO APA)
% -------------------------------------------------------------------------
\usepackage{booktabs} % Líneas horizontales profesionales (\toprule, \midrule)
\usepackage{caption}  % Control total de los títulos

% Configuración común para que diga "Tabla X" en negrita y Título en Cursiva abajo
\DeclareCaptionLabelSeparator*{newline}{\\} % Define el salto de línea

% -- TABLAS --
\captionsetup[table]{
    name=Tabla,
    labelsep=newline,          % Salto de línea entre número y título
    justification=raggedright, % Alineado a la izquierda
    singlelinecheck=false,     % Forza alineación izquierda
    labelfont=bf,              % "Tabla X" en Negrita
    textfont=it,               % "Título" en Cursiva
    skip=0.5em                 % Espacio antes de la tabla
}

% -- FIGURAS --
\captionsetup[figure]{
    name=Figura,
    labelsep=newline,
    justification=raggedright,
    singlelinecheck=false,
    labelfont=bf,
    textfont=it,
    skip=0.5em
}

% -------------------------------------------------------------------------
% 6. CONFIGURACIÓN DE ÍNDICES (TOC, LOT, LOF)
% -------------------------------------------------------------------------
\usepackage{tocloft}

% Elimina la numeración automática (1, 1.1) del índice y del texto
\setcounter{secnumdepth}{0}

% -- A) TABLA DE CONTENIDO --
\renewcommand{\contentsname}{Tabla de contenido}
\renewcommand{\cfttoctitlefont}{\hfill\Large\bfseries} % Título centrado
\renewcommand{\cftaftertoctitle}{\hfill}

% -- B) ÍNDICE DE TABLAS --
\renewcommand{\listtablename}{Índice de tablas}
\renewcommand{\cftlottitlefont}{\hfill\Large\bfseries}
\renewcommand{\cftafterlottitle}{\hfill}
% Prefijo: Agrega la palabra "Tabla" antes del número en el índice
\renewcommand{\cfttabpresnum}{Tabla }
\renewcommand{\cfttabaftersnum}{.}
\setlength{\cfttabnumwidth}{2.5cm} % Espacio reservado para "Tabla X."

% -- C) ÍNDICE DE FIGURAS --
\renewcommand{\listfigurename}{Índice de figuras}
\renewcommand{\cftloftitlefont}{\hfill\Large\bfseries}
\renewcommand{\cftafterloftitle}{\hfill}
% Prefijo: Agrega la palabra "Figura" antes del número en el índice
\renewcommand{\cftfigpresnum}{Figura }
\renewcommand{\cftfigaftersnum}{.}
\setlength{\cftfignumwidth}{2.5cm}

% -- D) Ajustes Visuales --
% Define sangrías escalonadas para el índice
\setlength{\cftsecindent}{0cm}
\setlength{\cftsubsecindent}{1.27cm}
\setlength{\cftsubsubsecindent}{2.54cm}
% Puntos suspensivos (...) guía hasta el número de página
\renewcommand{\cftsecleader}{\cftdotfill{\cftdotsep}}

% -------------------------------------------------------------------------
% 7. OTROS PAQUETES ÚTILES
% -------------------------------------------------------------------------
\usepackage{pdflscape}                   % Páginas en horizontal (Landscape)
\usepackage{amsmath,amsfonts,amssymb}    % Matemáticas avanzadas
\usepackage{graphicx}                    % Inserción de imágenes
\usepackage{float}                       % Para usar [H] (posición fija)
\usepackage{csquotes}                    % Comillas inteligentes

% Configuración de Hipervínculos (Negro para texto, Azul para URL)
\usepackage{hyperref}
\hypersetup{colorlinks=true,linkcolor=black,citecolor=black,urlcolor=blue}

% -------------------------------------------------------------------------
% 8. BIBLIOGRAFÍA
% -------------------------------------------------------------------------
\usepackage[natbibapa]{apacite} % Paquete oficial para citas APA
\bibliographystyle{apacite}

% =========================================================================
%                          INICIO DEL DOCUMENTO
% =========================================================================
\begin{document}

% --- PASO 1: CARÁTULA (Sin número de página) ---
\pagenumbering{gobble}    % Apaga la numeración
\newgeometry{margin=3cm}  % Márgenes estéticos para portada
% caratula.tex
\begin{titlepage}
    \centering
    
    % --- ENCABEZADO UNMSM ---
    {\bfseries\large UNIVERSIDAD NACIONAL MAYOR DE SAN MARCOS\par}
    {\scshape\small Universidad del Perú, Decana de América\par}
    \vspace{0.5cm}
    {\bfseries\large FACULTAD DE CIENCIAS ECONÓMICAS\par}
    \vspace{0.5cm}
    {\scshape\large Escuela Profesional de Economía Internacional\par}
    
    \vspace{1.5cm}
    
    % --- LOGO (Opcional) ---
    % Si no tienes la imagen aún, mantén esta línea comentada con %
    \includegraphics[width=4cm]{unmsm insignia.png} 
    
    
    % --- TÍTULO ---
    {\bfseries\ Determinantes de la victimización por extorsión en microempresas
de Huaycán: un estudio de corte transversal con metodología mixta\par}

    \vspace{3cm}
    %--profesor---
    {\large \textbf{Profesor:}}\par     
    \vspace{0.3cm}
    {\Large  Dr. Barrientos Felipa Pedro \par}
    
    \vspace{0.3cm}

    % --- AUTOR ---
    {\large \textbf{Presentado por:}}\par
    \vspace{0.3cm}
    {\Large Vargas Curisinche ULises Moisés \par}
    
    \vfill
    
    % --- PIE DE PÁGINA ---
    {\large Lima, Perú\par}
    {\large 2026}
    
\end{titlepage}        % Archivo externo 'Caratula.tex'
\restoregeometry          % Vuelve a márgenes APA (2.54cm)
\clearpage                % Salto de página

% --- PASO 2: ÍNDICES (Numeración Romana opcional o vacía) ---
% \pagenumbering{roman}   % Descomentar si la universidad pide romanos aquí

% A) Índice General
\tableofcontents
\newpage

% B) Índice de Tablas (Se agrega manualmente al TOC)
\addcontentsline{toc}{section}{Índice de tablas}
\listoftables

% C) Índice de Figuras (Se agrega manualmente al TOC)
\addcontentsline{toc}{section}{Índice de figuras}
\listoffigures
\newpage

% --- PASO 3: CUERPO DEL TRABAJO (Numeración Arábiga Oficial) ---
\pagenumbering{arabic}
\setcounter{page}{1} % La Introducción será la Página 1

% -------------------------------------------------------------------------
% EJEMPLO: ESTRUCTURA DE SECCIONES
% -------------------------------------------------------------------------
\section{Introducción}
Aquí comienza el texto principal. Observe que el título está centrado y en negrita (Nivel 1). El texto tiene sangría de 1.27 cm y está a doble espacio.

Para citar, LaTeX con `apacite` lo hace fácil:
\begin{itemize}
    \item Cita narrativa: \citet{Enders2014} afirma que...
    \item Cita parentética: Se ha encontrado evidencia \citep{Wooldridge2012}.
\end{itemize}

\section{Metodología}
Presentamos el modelo econométrico. Las ecuaciones se numeran a la derecha:
\begin{equation}
    Y_t = \beta_0 + \beta_1 X_{1t} + \mu_t
\end{equation}

% -------------------------------------------------------------------------
% EJEMPLO: CÓMO INSERTAR UNA TABLA APA
% -------------------------------------------------------------------------
\section{Resultados}

A continuación, se presenta la Tabla \ref{tab:ejemplo_regresion}. Note que solo se usan líneas horizontales principales.

\begin{table}[h] % [h] intenta ponerla 'aquí', [t] arriba, [b] abajo
    \centering
    
    % 1. TÍTULO (Configurado en preámbulo: Negrita arriba, Cursiva abajo)
    \caption{Resultados del análisis de regresión de la brecha salarial}
    \label{tab:ejemplo_regresion}
    
    % 2. TABLA (Use booktabs: toprule, midrule, bottomrule)
    \begin{tabular}{lccc}
        \toprule
        Variable & Coeficiente & Error Estándar & P-Valor \\
        \midrule
        Intercepto       & 15.40 & 2.30 & .001 \\
        Educación (Años) & 0.85  & 0.12 & .000 \\
        Género (Dummy)   & -1.20 & 0.45 & .032 \\
        \bottomrule
    \end{tabular}
    
    % 3. NOTA (Espacio pequeño + Nota explicativa)
    \vspace{0.1cm}
    \begin{minipage}{\linewidth}
        \footnotesize % Fuente ligeramente menor
        \textit{Nota.} Datos de la ENAHO 2024. * p<.05.
    \end{minipage}
\end{table}

% -------------------------------------------------------------------------
% EJEMPLO: CÓMO INSERTAR UNA FIGURA APA
% -------------------------------------------------------------------------
La Figura \ref{fig:ejemplo_grafico} muestra la estructura correcta.

\begin{figure}[H] % [H] obliga a la figura a quedarse EXACTAMENTE aquí
    \centering % Centra la imagen, no use el entorno 'center'
    
    % 1. TÍTULO
    \caption{Evolución de la victimización por extorsión (2015-2025)}
    \label{fig:ejemplo_grafico}
    
    % 2. IMAGEN
    % Asegúrese de que su gráfico (R/Stata) ya incluya la leyenda interna.
    \includegraphics[width=0.6\textwidth]{unmsm insignia.png} % Ejemplo
    
    % 3. NOTA
    \vspace{0.2cm}
    \begin{minipage}{0.8\textwidth} % Ancho ajustado a la imagen
        \footnotesize
        \textit{Nota.} La línea azul representa empresas formales. Fuente: INEI.
    \end{minipage}
\end{figure}

% -------------------------------------------------------------------------
% EJEMPLO: TABLA EN HORIZONTAL (LANDSCAPE)
% -------------------------------------------------------------------------
\newpage
\begin{landscape}
    \begin{table}[p]
        \centering
        \caption{Comparación de múltiples modelos (Formato Horizontal)}
        \label{tab:tabla_ancha}
        \begin{tabular}{lccccc}
            \toprule
            Variable & Mod 1 & Mod 2 & Mod 3 & Mod 4 & Mod 5 \\
            \midrule
            Intercepto & 2.45 & 2.10 & 1.95 & 1.80 & 1.50 \\
            $R^2$      & 0.45 & 0.52 & 0.58 & 0.60 & 0.65 \\
            \bottomrule
        \end{tabular}
        \vspace{0.1cm}
        \begin{minipage}{\linewidth}
            \footnotesize \textit{Nota.} Elaboración propia.
        \end{minipage}
    \end{table}
\end{landscape}

% -------------------------------------------------------------------------
% BIBLIOGRAFÍA
% -------------------------------------------------------------------------
\newpage
\bibliography{bibliografia} % Nombre del archivo .bib

\end{document}