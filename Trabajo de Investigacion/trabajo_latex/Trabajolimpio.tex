% =========================================================================
%      MANUAL Y PLANTILLA DE LATEX PARA NORMAS APA 7ma EDICIÓN
%                   Adaptado para la UNMSM
% =========================================================================

\documentclass[12pt, a4paper]{article}

% -------------------------------------------------------------------------
% 1. CODIFICACIÓN E IDIOMA
% -------------------------------------------------------------------------
\usepackage[utf8]{inputenc} % Permite tildes y ñ directamente
\usepackage[T1]{fontenc}    % Correcta codificación de salida
\usepackage[spanish, es-tabla, es-nodecimaldot]{babel}
% es-tabla: Cambia la etiqueta "Cuadro" por "Tabla" (Norma APA).
% es-nodecimaldot: Usa punto (.) para decimales (10.5) en vez de coma (10,5).

% -------------------------------------------------------------------------
% 2. FORMATO DE PÁGINA Y TEXTO (APA BÁSICO)
% -------------------------------------------------------------------------
% Márgenes: 1 pulgada (2.54 cm) en todos los lados.
\usepackage[margin=2.54cm]{geometry}

% Fuente: Times New Roman (Texto y Matemáticas).
\usepackage{mathptmx}

% Espaciado: Doble espacio estricto en todo el documento.
\usepackage{setspace}
\doublespacing

% Sangría: 1.27 cm (1/2 pulgada) en la primera línea de cada párrafo.
\setlength{\parindent}{1.27cm}

% -------------------------------------------------------------------------
% 3. ENCABEZADO Y NUMERACIÓN DE PÁGINA
% -------------------------------------------------------------------------
\usepackage{fancyhdr}
\setlength{\headheight}{15.2pt} % Ajuste técnico para evitar errores
\pagestyle{fancy}
\fancyhf{} % Limpia encabezados y pies por defecto

% LADO IZQUIERDO: Título Corto (Running Head)
% Nota: Debe ser en MAYÚSCULAS y máximo 50 caracteres.
\fancyhead[L]{DETERMINANTES DE LA EXTORSIÓN EN HUAYCÁN

% LADO DERECHO: Número de página
\fancyhead[R]{\thepage}

% Elimina la línea horizontal del encabezado (APA no usa líneas decorativas)
\renewcommand{\headrulewidth}{0pt}

% -------------------------------------------------------------------------
% 4. JERARQUÍA DE TÍTULOS (APA 7 - 5 NIVELES)
% -------------------------------------------------------------------------
\usepackage{titlesec}

% Nivel 1: Centrado, Negrita (Ej. \section{Introducción})
\titleformat{\section}[block]{\centering\normalfont\normalsize\bfseries}{}{0em}{}

% Nivel 2: Izquierda, Negrita (Ej. \subsection{Subtema})
\titleformat{\subsection}[block]{\raggedright\normalfont\normalsize\bfseries}{}{0em}{}

% Nivel 3: Izquierda, Negrita, Cursiva (Ej. \subsubsection{Detalle})
\titleformat{\subsubsection}[block]{\raggedright\normalfont\normalsize\bfseries\itshape}{}{0em}{}

% Nivel 4: Sangría, Negrita, Punto final. Texto sigue en la línea.
\titleformat{\paragraph}[runin]{\normalfont\normalsize\bfseries}{}{0em}{}[.]
\titlespacing{\paragraph}{1.27cm}{0pt}{1em}

% Nivel 5: Sangría, Negrita, Cursiva, Punto final. Texto sigue en la línea.
\titleformat{\subparagraph}[runin]{\normalfont\normalsize\bfseries\itshape}{}{0em}{}[.]
\titlespacing{\subparagraph}{1.27cm}{0pt}{1em}

% -------------------------------------------------------------------------
% 5. CONFIGURACIÓN DE TABLAS Y FIGURAS (ESTILO APA)
% -------------------------------------------------------------------------
\usepackage{booktabs} % Líneas horizontales profesionales (\toprule, \midrule)
\usepackage{caption}  % Control total de los títulos

% Configuración común para que diga "Tabla X" en negrita y Título en Cursiva abajo
\DeclareCaptionLabelSeparator*{newline}{\\} % Define el salto de línea

% -- TABLAS --
\captionsetup[table]{
    name=Tabla,
    labelsep=newline,          % Salto de línea entre número y título
    justification=raggedright, % Alineado a la izquierda
    singlelinecheck=false,     % Forza alineación izquierda
    labelfont=bf,              % "Tabla X" en Negrita
    textfont=it,               % "Título" en Cursiva
    skip=0.5em                 % Espacio antes de la tabla
}

% -- FIGURAS --
\captionsetup[figure]{
    name=Figura,
    labelsep=newline,
    justification=raggedright,
    singlelinecheck=false,
    labelfont=bf,
    textfont=it,
    skip=0.5em
}

% -------------------------------------------------------------------------
% 6. CONFIGURACIÓN DE ÍNDICES (TOC, LOT, LOF)
% -------------------------------------------------------------------------
\usepackage{tocloft}

% Elimina la numeración automática (1, 1.1) del índice y del texto
\setcounter{secnumdepth}{0}

% -- A) TABLA DE CONTENIDO --
\renewcommand{\contentsname}{Tabla de contenido}
\renewcommand{\cfttoctitlefont}{\hfill\Large\bfseries} % Título centrado
\renewcommand{\cftaftertoctitle}{\hfill}

% -- B) ÍNDICE DE TABLAS --
\renewcommand{\listtablename}{Índice de tablas}
\renewcommand{\cftlottitlefont}{\hfill\Large\bfseries}
\renewcommand{\cftafterlottitle}{\hfill}
% Prefijo: Agrega la palabra "Tabla" antes del número en el índice
\renewcommand{\cfttabpresnum}{Tabla }
\renewcommand{\cfttabaftersnum}{.}
\setlength{\cfttabnumwidth}{2.5cm} % Espacio reservado para "Tabla X."

% -- C) ÍNDICE DE FIGURAS --
\renewcommand{\listfigurename}{Índice de figuras}
\renewcommand{\cftloftitlefont}{\hfill\Large\bfseries}
\renewcommand{\cftafterloftitle}{\hfill}
% Prefijo: Agrega la palabra "Figura" antes del número en el índice
\renewcommand{\cftfigpresnum}{Figura }
\renewcommand{\cftfigaftersnum}{.}
\setlength{\cftfignumwidth}{2.5cm}

% -- D) Ajustes Visuales --
% Define sangrías escalonadas para el índice
\setlength{\cftsecindent}{0cm}
\setlength{\cftsubsecindent}{1.27cm}
\setlength{\cftsubsubsecindent}{2.54cm}
% Puntos suspensivos (...) guía hasta el número de página
\renewcommand{\cftsecleader}{\cftdotfill{\cftdotsep}}

% -------------------------------------------------------------------------
% 7. OTROS PAQUETES ÚTILES
% -------------------------------------------------------------------------
\usepackage{pdflscape}                   % Páginas en horizontal (Landscape)
\usepackage{amsmath,amsfonts,amssymb}    % Matemáticas avanzadas
\usepackage{graphicx}                    % Inserción de imágenes
\usepackage{float}                       % Para usar [H] (posición fija)
\usepackage{csquotes}                    % Comillas inteligentes

% Configuración de Hipervínculos (Negro para texto, Azul para URL)
\usepackage{hyperref}
\hypersetup{colorlinks=true,linkcolor=black,citecolor=black,urlcolor=black}

% -------------------------------------------------------------------------
% 8. BIBLIOGRAFÍA
% -------------------------------------------------------------------------
\usepackage[natbibapa]{apacite} % Paquete oficial para citas APA
\bibliographystyle{apacite}

% =========================================================================
%                          INICIO DEL DOCUMENTO
% =========================================================================
\begin{document}

% --- PASO 1: CARÁTULA (Sin número de página) ---
\pagenumbering{gobble}    % Apaga la numeración
\newgeometry{margin=3cm}  % Márgenes estéticos para portada
% caratula.tex
\begin{titlepage}
    \centering
    
    % --- ENCABEZADO UNMSM ---
    {\bfseries\large UNIVERSIDAD NACIONAL MAYOR DE SAN MARCOS\par}
    {\scshape\small Universidad del Perú, Decana de América\par}
    \vspace{0.5cm}
    {\bfseries\large FACULTAD DE CIENCIAS ECONÓMICAS\par}
    \vspace{0.5cm}
    {\scshape\large Escuela Profesional de Economía Internacional\par}
    
    \vspace{1.5cm}
    
    % --- LOGO (Opcional) ---
    % Si no tienes la imagen aún, mantén esta línea comentada con %
    \includegraphics[width=4cm]{unmsm insignia.png} 
    
    
    % --- TÍTULO ---
    {\bfseries\ Determinantes de la victimización por extorsión en microempresas
de Huaycán: un estudio de corte transversal con metodología mixta\par}

    \vspace{3cm}
    %--profesor---
    {\large \textbf{Profesor:}}\par     
    \vspace{0.3cm}
    {\Large  Dr. Barrientos Felipa Pedro \par}
    
    \vspace{0.3cm}

    % --- AUTOR ---
    {\large \textbf{Presentado por:}}\par
    \vspace{0.3cm}
    {\Large Vargas Curisinche ULises Moisés \par}
    
    \vfill
    
    % --- PIE DE PÁGINA ---
    {\large Lima, Perú\par}
    {\large 2026}
    
\end{titlepage}        % Archivo externo 'Caratula.tex'
\restoregeometry          % Vuelve a márgenes APA (2.54cm)
\clearpage                % Salto de página

% --- PASO 2: ÍNDICES (Numeración Romana opcional o vacía) ---
% \pagenumbering{roman}   % Descomentar si la universidad pide romanos aquí

% A) Índice General
\tableofcontents
\newpage

% B) Índice de Tablas (Se agrega manualmente al TOC)
\addcontentsline{toc}{section}{Índice de tablas}
\listoftables

% C) Índice de Figuras (Se agrega manualmente al TOC)
\addcontentsline{toc}{section}{Índice de figuras}
\listoffigures
\newpage

% --- PASO 3: CUERPO DEL TRABAJO (Numeración Arábiga Oficial) ---
\pagenumbering{arabic}
\setcounter{page}{1} % La Introducción será la Página 1

% -------------------------------------------------------------------------
% EJEMPLO: ESTRUCTURA DE SECCIONES
% -------------------------------------------------------------------------
% -------------------------------------------------------------------------
% EJEMPLO: ESTRUCTURA DE SECCIONES
% -------------------------------------------------------------------------
\section{JUSTIFICACIÓN}

La presente investigación pretende llenar un vacío en la literatura sobre la economía del crimen en zonas periurbanas de Lima, donde la informalidad laboral y empresarial genera vulnerabilidades específicas frente a la violencia. Si bien existen estudios sobre victimización por extorsión que identifican factores de cumplimiento como el miedo y la percepción de impunidad \citep{EstevezSoto2021}, pocos analizan cómo la heterogeneidad espacial (altitud y zonificación) y la precariedad institucional interactúan para determinar la probabilidad de extorsión en microempresas. El estudio contrastará la Teoría de las Actividades Rutinarias y la Elección Racional \citep{Schelling1971}, postulando que en Huaycán el crimen no es aleatorio, sino que responde a una lógica de mercado segmentado donde la falta de presencia estatal efectiva en zonas altas (Zona Z) y la aglomeración en zonas comerciales bajas (Zona A) crean incentivos diferenciados para las bandas criminales. Asimismo, se busca validar si la informalidad actúa como barrera de entrada o factor de riesgo ante la \enquote{gobernanza criminal}, concepto apoyado en la evidencia reciente sobre estrategias de pequeños negocios para gestionar extorsión en contextos de violencia \citep{Bull2024, GIATOC2019}.

La relevancia social de este estudio es crítica dado el contexto de inseguridad ciudadana que afecta a Lima Este. Reportes recientes señalan que la extorsión ha crecido un 478\% en los últimos cinco años a nivel nacional, afectando gravemente la continuidad de negocios y del sector transporte \citep{PuntoEdu2025}, con impactos económicos estimados en miles de millones de soles anuales por costos directos e indirectos de la violencia \citep{Jaitman2017}. Esta investigación visibilizará la \enquote{cifra negra} de la extorsión en la Comunidad Urbana Autogestionaria de Huaycán, exponiendo que la baja tasa de denuncias no responde solo al miedo, sino a una ruptura del contrato social y la desconfianza en la Policía Nacional \citep{Mininter2024, BancoMundial2021}. Los resultados permitirán a las juntas vecinales y comunidades empresariales diseñar estrategias de autoprotección más efectivas ante la inacción estatal, considerando los efectos de la extorsión en la entrada y salida emprendedora \citep{Estefan2025}.

El aporte principal reside en la construcción de un instrumento ad-hoc que supera las limitaciones de las encuestas nacionales, las cuales no capturan la micro-geografía de los conos de Lima ni los costos diferenciados de la violencia en microempresas informales. Se propone un diseño mixto: cuantitativamente, se aplicará un modelo logístico (Logit) incorporando variables espaciales (altitud) y de aglomeración, alineado con análisis de costos del crimen en la región \citep{Jaitman2017}; cualitativamente, se documentarán los mecanismos informales de resolución de conflictos ante la fallida respuesta policial, inspirado en estrategias de adaptación empresarial frente a extorsión \citep{Bull2024}. Este enfoque dual permitirá medir no solo la probabilidad de ser víctima, sino también entender la \enquote{economía del silencio} que impera en la zona, contribuyendo a la evidencia sobre impactos laborales y empresariales de la inseguridad \citep{BancoMundial2021}.


% ==========================================
% FORMULACIÓN DEL PROBLEMA
% ==========================================
\section{FORMULACIÓN DEL PROBLEMA}

\subsection{Problema Principal}
¿De qué manera los determinantes espaciales e institucionales influyen en la probabilidad de victimización por extorsión en las microempresas de la Comunidad Urbana Autogestionaria de Huaycán, durante el año 2025?

\subsection{Problemas Específicos}
\begin{enumerate}
    \item ¿Existe una relación significativa entre la ubicación geográfica altitudinal (zonas bajas vs. zonas altas) y la incidencia de extorsión en los negocios locales?
    \item ¿Cómo influye la percepción de ineficacia policial y las barreras institucionales de denuncia en la vulnerabilidad de los microempresarios frente al cobro de cupos?
    \item ¿En qué medida la aglomeración comercial (ubicación en mercados o galerías frente a ubicación aislada) actúa como factor diferenciador en los mecanismos de cobro extorsivo?
\end{enumerate}

% ==========================================
% OBJETIVOS DE LA INVESTIGACIÓN
% ==========================================
\section{OBJETIVOS DE LA INVESTIGACIÓN}

\subsection{Objetivo Principal}
Determinar la influencia de los factores espaciales e institucionales en la probabilidad de victimización por extorsión en las microempresas de la Comunidad Urbana Autogestionaria de Huaycán, 2025.

\subsection{Objetivos Específicos}
\begin{enumerate}
    \item Evaluar si la ubicación altitudinal (diferenciación entre zonas bajas y zonas altas) incrementa la probabilidad de ser víctima de extorsión.
    \item Analizar el efecto de la desconfianza institucional y la percepción de corrupción policial en la tasa de victimización efectiva de los microempresarios.
    \item Establecer la relación entre el tipo de aglomeración comercial (mercados vs. puerta a calle) y la modalidad de cobro de cupos.
\end{enumerate}
% -------------------------------------------------------------------------
% BIBLIOGRAFÍA
% -------------------------------------------------------------------------
\newpage
\bibliography{referencias} % Archivo externo 'referencias.bib'

\end{document}