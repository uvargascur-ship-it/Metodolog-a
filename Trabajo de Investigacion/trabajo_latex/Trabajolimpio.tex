\documentclass[12pt, a4paper]{article}

% --- PAQUETES DE IDIOMA ---
\usepackage[utf8]{inputenc}
\usepackage[T1]{fontenc}
\usepackage[spanish, es-tabla, es-nodecimaldot]{babel}

% --- FORMATO APA (MANUAL) ---
% 1. Márgenes: 2.54 cm (1 pulgada) en todos los lados
\usepackage[margin=2.54cm]{geometry}

% 2. Fuente: Times New Roman (Texto y Matemáticas)
\usepackage{mathptmx}

% 3. Espaciado: Doble espacio en todo el documento
\usepackage{setspace}
\doublespacing

% 4. Sangría: 1.27 cm en primera línea de párrafo
\setlength{\parindent}{1.27cm}

% --- PAQUETES PARA ECONOMÍA ---
\usepackage{amsmath, amsfonts, amssymb} % Ecuaciones
\usepackage{graphicx}                   % Imágenes
\usepackage{booktabs}                   % Tablas estéticas
\usepackage{float}                      % Para usar [H] en figuras
\usepackage{csquotes}                   % Comillas correctas

% --- BIBLIOGRAFÍA APA ---
% Usamos 'apacite' que es el estándar para BibTeX
\usepackage[natbibapa]{apacite}
\bibliographystyle{apacite}

% --- HIPERVÍNCULOS ---
\usepackage{hyperref}
\hypersetup{
    colorlinks=true,
    linkcolor=black, % APA pide negro
    citecolor=black, % APA pide negro
    urlcolor=blue    % URLs azules para facilitar lectura digital
}

% ==========================================
%           INICIO DEL DOCUMENTO
% ==========================================
\begin{document}

% 1. Insertar Carátula (sin numeración de página)
\newgeometry{margin=3cm} % Ajuste opcional: márgenes un poco más grandes para la portada estética
% caratula.tex
\begin{titlepage}
    \centering
    
    % --- ENCABEZADO UNMSM ---
    {\bfseries\large UNIVERSIDAD NACIONAL MAYOR DE SAN MARCOS\par}
    {\scshape\small Universidad del Perú, Decana de América\par}
    \vspace{0.5cm}
    {\bfseries\large FACULTAD DE CIENCIAS ECONÓMICAS\par}
    \vspace{0.5cm}
    {\scshape\large Escuela Profesional de Economía Internacional\par}
    
    \vspace{1.5cm}
    
    % --- LOGO (Opcional) ---
    % Si no tienes la imagen aún, mantén esta línea comentada con %
    \includegraphics[width=4cm]{unmsm insignia.png} 
    
    
    % --- TÍTULO ---
    {\bfseries\ Determinantes de la victimización por extorsión en microempresas
de Huaycán: un estudio de corte transversal con metodología mixta\par}

    \vspace{3cm}
    %--profesor---
    {\large \textbf{Profesor:}}\par     
    \vspace{0.3cm}
    {\Large  Dr. Barrientos Felipa Pedro \par}
    
    \vspace{0.3cm}

    % --- AUTOR ---
    {\large \textbf{Presentado por:}}\par
    \vspace{0.3cm}
    {\Large Vargas Curisinche ULises Moisés \par}
    
    \vfill
    
    % --- PIE DE PÁGINA ---
    {\large Lima, Perú\par}
    {\large 2026}
    
\end{titlepage}
\restoregeometry % Vuelve a márgenes APA (2.54cm)

% 2. Cuerpo del Trabajo
\setcounter{page}{1} % Empezamos a contar páginas aquí

\section{Introducción}
Este documento está configurado para seguir las normas APA manteniendo la flexibilidad de LaTeX. La fuente es Times New Roman de 12 puntos y el interlineado es doble.

Para citar tus fuentes de economía, usas el comando estándar. Por ejemplo, \cite{Enders2014} señala que las series de tiempo son fundamentales. También puedes citar al final de la frase \citep{Wooldridge2012}.

\section{Metodología}
En esta sección presentamos el modelo econométrico. Observa que las ecuaciones se numeran automáticamente:

\begin{equation}
    Y_t = \beta_0 + \beta_1 X_{1t} + \beta_2 X_{2t} + \mu_t
\end{equation}

Donde $Y_t$ representa la variable dependiente en el tiempo $t$.

\section{Resultados Preliminares}
Aquí puedes insertar tus tablas o gráficos exportados de R o Stata.

\section{Conclusiones}
Tus conclusiones finales sobre la investigación de verano.

% 3. Bibliografía
\newpage
% 3. Bibliografía
\bibliography{referencias}

\end{document}